\documentclass[handout]{beamer}
\usecolortheme[RGB={153,0,0}]{structure}
\usetheme{Rochester}
%\useinnertheme{rounded}
\setbeamertemplate{blocks}[rounded][shadow=true]
\useinnertheme{rectangles}
\usepackage[utf8]{inputenc}
\usepackage{idrislang}
\usepackage{listings}
\usepackage{natbib}
\usepackage{bibentry}
\usepackage{color}
\usepackage[spanish]{babel}
\usepackage{amssymb,latexsym}
\usepackage{xcolor}
\usepackage{hyperref}
\usepackage{alltt}

\setbeamertemplate{navigation symbols}{}
%\bibliography{Erlang}

\lstdefinestyle{idris-style}{
  belowcaptionskip=1\baselineskip,
  breaklines=true,
  %%frame=L,
  xleftmargin=\parindent,
  language=idris,
  showstringspaces=false,
  basicstyle=\footnotesize\ttfamily\color{blue},
  keywordstyle=\bfseries\color{green!40!black},
  commentstyle=\itshape\color{purple!40!black},
  identifierstyle=\color{black},
  stringstyle    =\color{red},
}

\lstset{style=idris-style,escapeinside=@@}%

\title{\idris}
\subtitle{A language with dependent types}
\author{Alejandro Gómez-Londoño}
\date{31th March, 2014}
\institute{EAFIT University}

\begin{document}
\begin{frame}
  \titlepage
\end{frame}

\begin{frame}{What is \idris}
  \begin{quotation}
    “What if Haskell had full dependent types?”
    \footnote[frame,1]{ Edwin Brady (2013). Idris, a general-purpose
      dependently typed programming language: Design and
      implementation. Journal of Functional Programming, 23, pp
      552-593.}
  \end{quotation}
\end{frame}

\begin{frame}{\idris features}
  \begin{itemize}
    \item Full dependent types
    \item Type classes
    \item \texttt{where} clauses, \texttt{do} notation,\texttt{let} bindings
    \item Monad comprehensions
    \item Totality checking
    \item Cumulative universes
    \item Tactic based theorem proving
    \item Simple foreign function interface (to C)
  \end{itemize}
\end{frame}

\begin{frame}[fragile]{\idris}{Basic Types}
\begin{alltt}

  {\color{red}{Z}}     : {\color{blue}{Nat}}
  {\color{red}{50}}    : {\color{blue}{Integer}}
  {\color{red}{1.23}}  : {\color{blue}{Float}}
  {\color{red}{True}}  : {\color{blue}{Bool}}\pause
  {\color{red}{'a'}}   : {\color{blue}{Char}}
  {\color{red}{"foo"}} : {\color{blue}{String}}\pause

  {\color{red}{[1,2,3]}} : {\color{blue}{List Integer}}
  {\color{red}{[1,2,3]}} : {\color{blue}{Vect 3 Integer}}
\end{alltt}
\end{frame}

\begin{frame}[fragile]{\idris}
  {Data Types\footnote[frame,1]
    {Programming in Idris: a tutorial, Edwin Brady January 2012}}
  \begin{lstlisting}
    data Nat = Z | S Nat @\pause@

    data Bool = True | False@\pause@

    infixr 10 ::
    data List a = Nil | (::) a (List a)@\pause@

    record Person : Type where
      MkPerson : (name : String) ->
                 (age  : Int)    -> Person
  \end{lstlisting}
\end{frame}

\begin{frame}[fragile]{\idris}
  {functions\footnote[frame,1]
    {Programming in Idris: a tutorial, Edwin Brady January 2012}}
  \begin{lstlisting}
    plus : Nat -> Nat -> Nat
    plus Z     y = y
    plus (S k) y = S (plus k y)@\pause@

    mult : Nat -> Nat -> Nat
    mult Z     y = Z
    mult (S k) y = plus y (mult k y)@\pause@

    fact : Nat -> Nat
    fact Z     = 1
    fact (S k) = (S k)*(fact k)
  \end{lstlisting}
\end{frame}

\begin{frame}[fragile]{\idris}
  {\texttt{do, where, let}\footnote[frame,1]
    {Programming in Idris: a tutorial, Edwin Brady January 2012}}
\begin{lstlisting}
  mirror : List a -> List a
  mirror xs = let xs' = reverse xs in
                  xs ++ xs'@\pause@

  even : Nat -> Bool
  even Z     = True
  even (S k) = odd k where
    odd Z     = False
    odd (S k) = even k@\pause@

  greet : IO ()
  greet = do 
    putStrLn "What is your name? "
    name <- getLine
    putStrLn ("Hello " ++ name)


\end{lstlisting}
\end{frame}

\begin{frame}{Dependent Types}{Definition}
  \begin{quotation}
    In conventional programming languages, there is a clear
    distinction between types and values... \\
    In a language with dependent types, however, the distinction is
    less clear. Dependent types allow types to ``depend'' on values -
    in other words, types are a first class language construct and can
    be manipulated like any other value.\footnote[frame,1]
    {Programming in Idris: a tutorial, Edwin Brady January 2012}
  \end{quotation}
\end{frame}

\begin{frame}[fragile]{Dependent Types}{Example on data types}
  \begin{lstlisting}
    data Vect : Nat -> Type -> Type where
      Nil  : Vect Z a
      (::) : a -> Vect k a -> Vect (S k) a @\pause@


    data VectSum : Nat -> Nat -> Type where
      Nil  : VectSum Z Z
      (::) : (b : Nat) ->
             VectSum k a ->
             VectSum (S k) (a + b)
    \end{lstlisting}
\end{frame}

\begin{frame}[fragile]{Dependent Types}{Example on functions}
  \begin{lstlisting}
    (++) : Vect n a -> Vect m a -> Vect (n + m) a
    (++) Nil       ys = ys
    (++) (x :: xs) ys = x :: xs ++ ys @\pause@


    vecHead : Vect n a -> so (n > 0) -> a
    vecHead (x :: xs) _ = x @\pause@


    vecHead' : Vect (S n) a -> a
    vecHead' (x :: xs) =  x
  \end{lstlisting}
\end{frame}

\begin{frame}[fragile]{Dependent Types}{Examples on Implicit Arguments}
  \begin{lstlisting}
  vectMap : (A : Type) -> ( B : Type)
            -> (A -> B)-> Vect n A -> Vect n B
  vectMap _  _  f Nil     = Nil
  vectMap t1 t2 f (x::xs) = 
    f x :: vectMap t1 t2 f xs @\pause@


  vectMap' : {A : Type} -> {B : Type}
             -> (A -> B)-> Vect n A -> Vect n B
  vectMap' f Nil     = Nil
  vectMap' f (x::xs) = f x :: vectMap' f xs @\pause@


  vectMap'' : (a -> b)-> Vect n a -> Vect n b
  vectMap'' f Nil     = Nil
  vectMap'' f (x::xs) = f x :: vectMap'' f xs @\pause@
  \end{lstlisting}
\end{frame}

\begin{frame}[fragile]{Theorem Proving}
  \begin{lstlisting}
    data (=) : a -> b -> Type where
      refl : x = x
  \end{lstlisting}
  \pause
  Now some examples...
\end{frame}
\begin{frame}{Theorem Proving}{commands and tactics
    \footnote[frame,1]
    {\idris-wiki,\url{https://github.com/idris-lang/Idris-dev/wiki/Manual}}}
  \begin{description}
  \item[compute] Normalizes all terms in the goal (note: does not
    normalize assumptions)
  \item[exact] Provide a term of the goal type directly
  \item[trivial] Satisfies the goal using an assumption that
    matches its type
  \item[intro] If your goal is an arrow, turns the left term into
    an assumption
  \item[intros] Exactly like intro, but it operates on all left
    terms at once
  \item[let] Introduces a new assumption; you may use current
    assumptions to define the new one
  \end{description}
\end{frame}


\begin{frame}{Theorem Proving}{commands and tactics
    \footnote[frame,1]
    {\idris-wiki,\url{https://github.com/idris-lang/Idris-dev/wiki/Manual}}}
  \begin{description}
  \item[rewrite] Takes an expression with an equality type (x =
    y), and replaces all instances of x in the goal with y. Is often
    useful in combination with 'sym'
  \item[state] Displays the current state of the proof
  \item[term] Displays the current proof term complete with its
    yet-to-be-filled holes
  \item[undo] Undoes the last tactic
  \item[qed] Once the interactive theorem prover tells you ``No
    more goals,'' you get to type this in celebration!
  \end{description}
\end{frame}
\end{document}
